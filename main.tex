% !TeX program = xelatex
\def\thelstlisting{}

%不需要区分奇偶页的请使用下面一行
\documentclass[a4paper,AutoFakeBold,oneside,12pt]{book}
%需要区分奇偶页的(即每一章第一页一定在奇数页上)请使用下面一行
%\documentclass[a4paper,AutoFakeBold,openright,12pt]{book}
\usepackage{BUPTthesisbachelor}
\usepackage{setspace}

% 标题区与正文区仅保留目录/标题框架;正文内容请自行填写
\definecolor{linkblue}{RGB}{26,102,178}
\hypersetup{
  colorlinks=true,
  linkcolor=linkblue,
  citecolor=linkblue,
  urlcolor=linkblue
}

%%%%%%%%%%%%%%%%%%%%%%%%% Begin Documents %%%%%%%%%%%%%%%%%%%%%%%%%%
\begin{document}

% 封面
% 约定:封面 PDF 放到 `docs/cover.pdf`
\blankmatter
\includepdf[pages=-]{docs/cover.pdf}

%%%%%%%%%%%%%%%%%%%%%%%%%%%%%%%%%%%%%%%%%%%%%%%%%%%%%%%%%%%%%%%%%%%%
%                                                                  %
%   Copyright (c) 2010 - 2011 Caspar Zhang <casparant@gmail.com>   %
%                                                                  %
%   This copyrighted material is made available to anyone wishing  %
%   to use, modify, copy, or redistribute it subject to the terms  %
%   and conditions of the GNU General Public License version 2.    %
%                                                                  %
%   This program is distributed in the hope that it will be        %
%   useful, but WITHOUT ANY WARRANTY; without even the implied     %
%   warranty of MERCHANTABILITY or FITNESS FOR A PARTICULAR        %
%   PURPOSE. See the GNU General Public License for more details.  %
%                                                                  %
%   You should have received a copy of the GNU General Public      %
%   License along with this program; if not, write to the Free     %
%   Software Foundation, Inc., 51 Franklin Street, Fifth Floor,    %
%   Boston, MA 02110-1301, USA.                                    %
%                                                                  %
%%%%%%%%%%%%%%%%%%%%%%%%%%%%%%%%%%%%%%%%%%%%%%%%%%%%%%%%%%%%%%%%%%%%

% 论文中文题目
\def\thesistitle{请在此填写中文题目}

% 论文英文题目
% 提示:英文摘要页的标题注意格式要求
\def\thesistitleen{Please input English title}

% Thank Words
\def\thankwords{}
    % Main items
%%%%%%%%%%%%%%%%%%%%%%%%%%%%%%%%%%%%%%%%%%%%%%%%%%%%%%%%%%%%%%%%%%%%
%                                                                  %
%   Copyright (c) 2010 - 2011 Caspar Zhang <casparant@gmail.com>   %
%                                                                  %
%   This copyrighted material is made available to anyone wishing  %
%   to use, modify, copy, or redistribute it subject to the terms  %
%   and conditions of the GNU General Public License version 2.    %
%                                                                  %
%   This program is distributed in the hope that it will be        %
%   useful, but WITHOUT ANY WARRANTY; without even the implied     %
%   warranty of MERCHANTABILITY or FITNESS FOR A PARTICULAR        %
%   PURPOSE. See the GNU General Public License for more details.  %
%                                                                  %
%   You should have received a copy of the GNU General Public      %
%   License along with this program; if not, write to the Free     %
%   Software Foundation, Inc., 51 Franklin Street, Fifth Floor,    %
%   Boston, MA 02110-1301, USA.                                    %
%                                                                  %
%%%%%%%%%%%%%%%%%%%%%%%%%%%%%%%%%%%%%%%%%%%%%%%%%%%%%%%%%%%%%%%%%%%%

% 中文摘要
\def\abstractzh{}

% 中文关键字
\def\abszhkeyone{}
\def\abszhkeytwo{}
\def\abszhkeythree{}
\def\abszhkeyfour{}
\def\abszhkeyfive{}

% ABSTRACT
\def\abstracten{}

% Key Words
\def\absenkeyone{}
\def\absenkeytwo{}
\def\absenkeythree{}
\def\absenkeyfour{}
\def\absenkeyfive{}
  % Abstract

% 目录
\fancypagestyle{plain}{\pagestyle{frontmatter}}
\frontmatter
\tableofcontents

% 正文
\mainmatter
\fancypagestyle{plain}{\pagestyle{mainmatter}}

%%%%%%%%%%%%%%%%%%%%%%%%%%%%% Main Area %%%%%%%%%%%%%%%%%%%%%%%%%%%%

\chapter{引言}

\section{背景介绍}
\subsection{矩阵理论与方法介绍}
\subsection{函数矩阵和矩阵函数介绍}
\subsection{线性代数方程组求解介绍}

\section{问题介绍}
\subsection{矩阵函数的求法问题介绍}
\subsection{矩阵分解的方法问题介绍}

\section{上述问题国内外研究成果介绍}
\subsection{矩阵函数的求法研究现状}
\subsection{矩阵分解方法研究现状}

\section{本论文工作简述}
\subsection{本论文对上述问题研究简述}
\subsection{本论文创新点或特点简述}
\subsection{本论文撰写结构简述}

\chapter{预备知识}

\section{欧式空间与线性变换}
\subsection{欧式空间与线性变换介绍}
\subsection{若尔当标准形的求解}
\subsection{欧式空间中线性变换的求法(解法参考课本例1.36和ppt)}

\section{向量范数与矩阵范数}
\subsection{向量范数介绍}
\subsection{矩阵范数介绍}
\subsection{矩阵可逆性条件、条件数和谱半径介绍}

\section{矩阵函数介绍}
\subsection{矩阵序列介绍}
\subsection{矩阵级数介绍}
\subsection{矩阵函数介绍 (参考课本3.3.1)}
\subsection{函数矩阵对矩阵的导数}

\chapter{矩阵函数的求法研究}

\section{待定系数法}
\subsection{待定系数法求矩阵函数的步骤推导}
\subsection{举例详细展示求法}

\section{数项级数求和法}
\subsection{数项级数求和法求矩阵函数的步骤推导}
\subsection{举例详细展示求法}

\section{对角形法}
\subsection{对角形法求矩阵函数的步骤推导}
\subsection{举例详细展示求法}

\section{若尔当标准形法}
\subsection{若尔当标准形法求矩阵函数的步骤推导}
\subsection{举例详细展示求法}

\section{本章小节}

\chapter{矩阵分解方法研究}

\section{矩阵的LU分解}
\subsection{矩阵LU分解的步骤推导}
\subsection{举例详细展示求法}

\section{矩阵的QR分解}
\subsection{矩阵QR分解的步骤推导}
\subsection{举例详细展示求法}

\section{矩阵的满秩分解}
\subsection{矩阵满秩分解的步骤推导}
\subsection{举例详细展示求法}

\section{矩阵的奇异值分解}
\subsection{矩阵奇异值分解的步骤推导}
\subsection{举例详细展示求法}

\section{利用矩阵分解求矩阵广义逆}
\subsection{矩阵广义逆介绍}
\subsection{利用矩阵满秩分解求矩阵广义逆}
\subsection{利用矩阵奇异值分解求矩阵广义逆}
\subsection{举例详细展示求法}

\section{本章小节}

\chapter{总结}

%%%%%%%%%%%%%%%%%%%%%%%%%%%%% References %%%%%%%%%%%%%%%%%%%%%%%%%%%%
\clearpage\phantomsection\addcontentsline{toc}{chapter}{参考文献}
\bibliographystyle{buptbachelor}
\refbodyfont{\bibliography{ref}}

\end{document}
